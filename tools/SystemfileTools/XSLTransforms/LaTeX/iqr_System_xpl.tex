\documentclass[a4paper, 11pt]{article}
\usepackage[T1]{fontenc}
\usepackage[english]{babel} %das Letzte ist default
\usepackage{paralist}
\usepackage{treelist}
\selectlanguage{english}

\begin{document}

\section{System TestSystem}

%%%---------------
\subsection{Process PrcM1}


\subsubsection{Module UserModule1}
Process PrcM1 has a UserModule \emph{UserModule1} which resides in
\verb!blabla.so!.

\begin{center}
\begin{tabular}{l|c|c}
Name & Type & Value\\\hline
ModuleParameter1 & short & 0.1100\\
ModuleParameter2 & string & value\\
ModuleParameter3 & int & 4510\\
\end{tabular}
\end{center}

\subsubsection{Group GroupA}
\label{sec:Group_g0}

\paragraph*{Geometry}

The geometry of group GroupA is \emph{sparse}, and defined
by the following points:
\begin{compactitem}
   \item point: (10, 10)
   \item point: (20, 20)
\end{compactitem}

\paragraph*{Cell Type}
The cell type of the group is \emph{"Sigmoid"}.

%%%---------------
\subsection{Process PrcM2}


\subsubsection{Module ModulePrcM2}
Process PrcM2 has a UserModule \emph{ModulePrcM2} which resides in
\verb!defaultCDATA85!.

\begin{center}
\begin{tabular}{l|c|c}
Name & Type & Value\\\hline
ModuleParam0 & string & string\\
ModuleParam1 & int & 100\\
ModuleParam2 & double & 10010.22\\
\end{tabular}
\end{center}

\subsubsection{Group Group0PrcM2}
\label{sec:Group_g1}

\paragraph*{Geometry}

The geometry of group Group0PrcM2 is a \emph{rectangle} with
$height= 12$, and $width= 17$.

\paragraph*{Cell Type}
The cell type of the group is \emph{"Sigmoid"}.
\paragraph*{Connections to other groups}
\begin{compactenum}
\item to group Group0PrcS3 (see \ref{sec:Group_g8})
\item to group Group0PrcS2 (see \ref{sec:Group_g5})
\item to group Group0PrcS1 (see \ref{sec:Group_g3})
\end{compactenum}

\subsubsection{Group Group1PrcM2}
\label{sec:Group_g2}

\paragraph*{Geometry}

The geometry of group Group1PrcM2 is \emph{hexagonal}, with
the following parameters
\begin{compactitem}
    \item height: defaultCDATA56
    \item width: defaultCDATA58
    \item offset: defaultCDATA59
    \item orientation: defaultCDATA57
\end{compactitem}

\paragraph*{Cell Type}
The cell type of the group is \emph{"User Defined"} called \emph{MyNeuronType} 
and is loaded from \verb!myneuron.o!.


The neuron has the following parameters:

\begin{center}
\begin{tabular}{l|c|c}
Name & Type & Value\\\hline
MyParam & int & 100\\
\end{tabular}
\end{center}

\paragraph*{Connections to other groups}
\begin{compactenum}
\item to group Group0PrcM2 (see \ref{sec:Group_g1})
\item to group GroupA (see \ref{sec:Group_g0})
\end{compactenum}


%%%---------------
\subsection{Process PrcS1}


\subsubsection{Group Group0PrcS1}
\label{sec:Group_g3}

\paragraph*{Geometry}

The geometry of group Group0PrcS1 is a \emph{rectangle} with
$height= 30$, and $width= 30$.

\paragraph*{Cell Type}
The cell type of the group is \emph{"User Defined"} called \emph{YAUDNT} 
and is loaded from \verb!myneuron2.o!.


The neuron has the following parameters:

\begin{center}
\begin{tabular}{l|c|c}
Name & Type & Value\\\hline
UDNParameter & int & 100\\
UDNParameter2 & short & 0.001\\
UDNParameter3 & string & blabla\\
\end{tabular}
\end{center}

\paragraph*{Connections to other groups}
\begin{compactenum}
\item to group Group0PrcM2 (see \ref{sec:Group_g1})
\item to group Group2PrcS2 (see \ref{sec:Group_g7})
\item to group Group0PrcS3 (see \ref{sec:Group_g8})
\end{compactenum}

\subsubsection{Group Group1PrcS1}
\label{sec:Group_g4}

\paragraph*{Geometry}

The geometry of group Group1PrcS1 is \emph{sparse}, and defined
by the following points:
\begin{compactitem}
   \item point: (0, 1)
\end{compactitem}

\paragraph*{Cell Type}
The cell type of the group is \emph{"Random Spike"}.

%%%---------------
\subsection{Process PrcS2}


\subsubsection{Group Group0PrcS2}
\label{sec:Group_g5}

\paragraph*{Geometry}

The geometry of group Group0PrcS2 is a \emph{rectangle} with
$height= 100$, and $width= 100$.

\paragraph*{Cell Type}
The cell type of the group is \emph{"Linear Threshold"}.
\paragraph*{Connections to other groups}
\begin{compactenum}
\item to group Group0PrcM2 (see \ref{sec:Group_g1})
\item to group Group1PrcM2 (see \ref{sec:Group_g2})
\item to group Group0PrcS1 (see \ref{sec:Group_g3})
\end{compactenum}

\subsubsection{Group Group1PrcS2}
\label{sec:Group_g6}

\paragraph*{Geometry}

The geometry of group Group1PrcS2 is \emph{sparse}, and defined
by the following points:
\begin{compactitem}
   \item point: (1, 2)
   \item point: (2, 3)
   \item point: (4, 5)
\end{compactitem}

\paragraph*{Cell Type}
The cell type of the group is \emph{"Integrate and Fire"}.
\paragraph*{Connections to other groups}
\begin{compactenum}
\item to group Group0PrcS3 (see \ref{sec:Group_g8})
\item to group Group2PrcS2 (see \ref{sec:Group_g7})
\end{compactenum}

\subsubsection{Group Group2PrcS2}
\label{sec:Group_g7}

\paragraph*{Geometry}

The geometry of group Group2PrcS2 is \emph{hexagonal}, with
the following parameters
\begin{compactitem}
    \item height: 200
    \item width: 300
    \item offset: 10
    \item orientation: horizontal
\end{compactitem}

\paragraph*{Cell Type}
The cell type of the group is \emph{"TanH"}.
\paragraph*{Connections to other groups}
\begin{compactenum}
\item to group Group0PrcM2 (see \ref{sec:Group_g1})
\end{compactenum}


%%%---------------
\subsection{Process PrcS3}


\subsubsection{Group Group0PrcS3}
\label{sec:Group_g8}

\paragraph*{Geometry}

The geometry of group Group0PrcS3 is \emph{sparse}, and defined
by the following points:
\begin{compactitem}
   \item point: (20, 2)
\end{compactitem}

\paragraph*{Cell Type}
The cell type of the group is \emph{"TanH"}.
\paragraph*{Connections to other groups}
\begin{compactenum}
\item to group Group1PrcS1 (see \ref{sec:Group_g4})
\end{compactenum}



\end{document}


%%% Local Variables:
%%% mode: latex
%%% TeX-master: t
%%% End:

